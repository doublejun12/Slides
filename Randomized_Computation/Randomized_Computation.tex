%%%%%%%%%%%%%%%%%%%%%%%%%%%%%%%%%%%%%%%%%
% Beamer Presentation
% LaTeX Template
% Version 1.0 (10/11/12)
%
% This template has been downloaded from:
% http://www.LaTeXTemplates.com
%
% License:
% CC BY-NC-SA 3.0 (http://creativecommons.org/licenses/by-nc-sa/3.0/)
%
%%%%%%%%%%%%%%%%%%%%%%%%%%%%%%%%%%%%%%%%%

%----------------------------------------------------------------------------------------
%	PACKAGES AND THEMES
%----------------------------------------------------------------------------------------

\documentclass{beamer}

\mode<presentation> {

% The Beamer class comes with a number of default slide themes
% which change the colors and layouts of slides. Below this is a list
% of all the themes, uncomment each in turn to see what they look like.

%\usetheme{default}
%\usetheme{AnnArbor}
%\usetheme{Antibes}
%\usetheme{Bergen}
%\usetheme{Berkeley}
%\usetheme{Berlin}
%\usetheme{Boadilla}
%\usetheme{CambridgeUS}
%\usetheme{Copenhagen}
%\usetheme{Darmstadt}
%\usetheme{Dresden}
%\usetheme{Frankfurt}
%\usetheme{Goettingen}
%\usetheme{Hannover}
%\usetheme{Ilmenau}
%\usetheme{JuanLesPins}
%\usetheme{Luebeck}
\usetheme{Madrid}
%\usetheme{Malmoe}
%\usetheme{Marburg}
%\usetheme{Montpellier}
%\usetheme{PaloAlto}
%\usetheme{Pittsburgh}
%\usetheme{Rochester}
%\usetheme{Singapore}
%\usetheme{Szeged}
%\usetheme{Warsaw}

% As well as themes, the Beamer class has a number of color themes
% for any slide theme. Uncomment each of these in turn to see how it
% changes the colors of your current slide theme.

%\usecolortheme{albatross}
%\usecolortheme{beaver}
%\usecolortheme{beetle}
%\usecolortheme{crane}
%\usecolortheme{dolphin}
%\usecolortheme{dove}
%\usecolortheme{fly}
%\usecolortheme{lily}
%\usecolortheme{orchid}
%\usecolortheme{rose}
%\usecolortheme{seagull}
%\usecolortheme{seahorse}
%\usecolortheme{whale}
%\usecolortheme{wolverine}

%\setbeamertemplate{footline} % To remove the footer line in all slides uncomment this line
%\setbeamertemplate{footline}[page number] % To replace the footer line in all slides with a simple slide count uncomment this line

%\setbeamertemplate{navigation symbols}{} % To remove the navigation symbols from the bottom of all slides uncomment this line
}

\usepackage{graphicx} % Allows including images
\usepackage{booktabs} % Allows the use of \toprule, \midrule and \bottomrule in tables
\usepackage{color}

%----------------------------------------------------------------------------------------
%	TITLE PAGE
%----------------------------------------------------------------------------------------

\title[Short title]{Randomized Computation} % The short title appears at the bottom of every slide, the full title is only on the title page

\author{Shuangjun Zhang} % Your name
\institute[FDU] % Your institution as it will appear on the bottom of every slide, may be shorthand to save space
{
Fudan University \\ % Your institution for the title page
\medskip
\textit{zhangsj17@fudan.edu.cn} % Your email address
}
\date{\today} % Date, can be changed to a custom date

\begin{document}

\begin{frame}
\titlepage % Print the title page as the first slide
\end{frame}

\begin{frame}
\frametitle{Overview} % Table of contents slide, comment this block out to remove it
\tableofcontents % Throughout your presentation, if you choose to use \section{} and \subsection{} commands, these will automatically be printed on this slide as an overview of your presentation
\end{frame}

%----------------------------------------------------------------------------------------
%	PRESENTATION SLIDES
%----------------------------------------------------------------------------------------

\section{Probabilistic Turing Machines}
%------------------------------------------------
\subsection{Polynomial Identity Testing}
\begin{frame}
\frametitle{Polynomial Identity Testing}
Let A(x), B(x) be polynomials(over R) of degrees n\\
\begin{itemize}
	\item {\color{red} Question}: Is $A(x) = B(x)$?
	\item {\color{red} Naive Idea}: Converting the two polynomials to their canonical forms($\Sigma_{i=0}^{n} a_{i}x^{i}$),two polynomials are equivalent iff all the coeficients in their canonical forms are equal.
	\item {\color{red} Another Idea}: Reduce problem to checking zero’s of a polynomial.
	$$
	A(x) = B(x) \Rightarrow A(x) - B(x) = 0
	$$
	\begin{itemize}
		\item If $A(x) = B(x)$, $\forall r \in R, A(r) - B(r) = 0$.  
		\item If $A(x) \neq B(x)$, $A(x)-B(x) = 0$ has at most n roots.
	\end{itemize}
\end{itemize}

\end{frame}
%------------------------------------------------
\begin{frame}
\frametitle{Polynomial Identity Testing}
	{\color{red} Algorithm PIT}:\\
	{\color{red} Input}: A(x), B(x), with maximum degree is n\\
	\begin{itemize}
		\item Choose $r \in \{1,2,...,100n\}$ at random;
		\item Computer A(r) and B(r);
		\item If $A(r) \neq B(r)$ {\color{red} Return} 0;
		\item else {\color{red} Return} 1;
	\end{itemize}
\end{frame}
%------------------------------------------------
\begin{frame}
\frametitle{Polynomial Identity Testing}
	{\color{red} Analysis:}
	\begin{itemize}
		\item If $\exists$ r, s.t. A(r) $\neq$ B(r), then A(x) $\neq$ B(x),that is 
		$$Pr[A(x) \neq B(x)|PIT() = 0] = 1$$
		\item If $A(x) = B(x)$,then $\forall$ r, $A(r) = B(r)$,that is
		$$Pr[PIT()=1|A(x) = B(x)] = 1$$
		\item If $A(x) \neq B(x)$,then with probability at most 1/100, $A(r) = B(r)$, that is 
		$$Pr[PIT()=1|A(x) \neq B(x)] \leq 1/100$$
	\end{itemize}
\end{frame}

%------------------------------------------------
\subsection{Probabilistic Turing Machines}
\begin{frame}
\frametitle{Probabilistic Turing Machines(PTM)}
\begin{itemize}
	\item So far, considered the following types of machines for input x:
	\begin{itemize}
		\item {\color{red} TM}: only one computation that either accepts or rejects
		\item {\color{red} NTM}: there are many sequences of computation, and x accepted if there exists a sequence that accepts

		\item {\color{red} ATM}: many sequences of computations, and x accepted based on the type of nodes ($\exists$ or $\forall$) along its path
		\item {\color{red} OTM}:a TM with access to an oracle
	\end{itemize}
	\item Another kind of machine: {\color{red} Probabilistic Turing Machine}
	\begin{itemize}
		\item Very similar to a TM , except at each step, M could toss a coin to get a random bit, and decide which transition rule to follow based on the outcome of the coin toss
	\end{itemize}
	\item A PTM takes input x and a random bit string r
	\item $Pr[M(x,r) = z]$: probability M outputs z
\end{itemize}
\end{frame}
%-------------------------------------------------------------
\subsection{Failure Probability}
\begin{frame}
\frametitle{Failure Probability}
A major aspect of randomized algorithms or probabilistic machines is that they may fail to produce the desired output with some specified failure probability.

\begin{itemize}
	\item One-sided error
	\begin{itemize}
		\item The machine may err only in one direction  i.e. either on ‘yes’ instances or on ‘no’ instances.

	\end{itemize}
	\item Two-sided error
	\begin{itemize}
		\item The machine may err in both directions i.e. it may result a ‘yes’ instance as a ‘no’ instance and vice versa.

	\end{itemize}
	\item Zero-sided error
	\begin{itemize}
		\item The algorithm never provides a wrong answer.
		\item But sometimes it returns 'don't know'
	\end{itemize}
\end{itemize}
\end{frame}
%------------------------------------------------------
\section{Randomized Complexity Classes}
\begin{frame}
\frametitle{Randomized Complexity Classes}
\begin{itemize}
	\item {\color{red} RP} (Randomized Polynomial time)
	\begin{itemize}
		\item One-sided error

	\end{itemize}
	\item {\color{red} coRP} (complement of RP )
	\begin{itemize}
		\item One-sided error
	\end{itemize}
	\item {\color{red} BPP} (Bounded-error Probabilistic Polynomial time)
	\begin{itemize}
		\item Two-sided error
	\end{itemize}
	\item {\color{red} ZPP} (Zero-error Probabilistic Polynomial time)
	\begin{itemize}
		\item Zero error
	\end{itemize}
\end{itemize}
\end{frame}
%------------------------------------------------------
\subsection{RP and coRP}
\begin{frame}
\frametitle{RP and coRP}
	\begin{definition}[RP]
		The complexity class RP is the class of all languages L for which there exists a polynomial PTM M such that

		$$
		x \in L \Rightarrow Pr[M(x) = 1] \geq 1/2
		$$
		$$
		x \notin L \Rightarrow Pr[M(x) = 0] = 1
		$$
	\end{definition}
	\begin{definition}[coRP]
		The complexity class coRP is the class of all languages L for which there exists a polynomial PTM M such that

		$$
		x \in L \Rightarrow Pr[M(x) = 1] = 1
		$$
		$$
		x \notin L \Rightarrow Pr[M(x) = 0] \geq 1/2
		$$
	\end{definition}
\end{frame}
%------------------------------------------------------
\begin{frame}
\frametitle{RP is in NP}
\begin{theorem}
	$$RP \subseteq NP$$
\end{theorem}
\begin{proof}
	\begin{itemize}
		\item Let L be a language in RP . Let M be a polynomial probabilistic Turing Machine that decides L.
		\item If x $\in$ L, then there exists a sequence of coin tosses r such that M accepts x with r as the random string.

		\item So we can consider r as a certificate instead of a random string. r can be verified in polynomial time by the same machine.
		\item If x $\notin$ L, then $Pr[M(x) = 1] = 0$. So there is no certificate r s.t. $M(x, r) = 1$.
		\item So, L $\in$ NP
	\end{itemize}
\end{proof}
\end{frame}
%------------------------------------------------------
\begin{frame}
\frametitle{Error Reduction for RP}
\begin{itemize}
	\item The constant 1/2 in the definition of RP can be replaced by any constant k, 0 < k < 1
	\item Since the error is one-sided, we can repeat the algorithm t times independently:
	\item Clearly, if x $\notin$ L, all t runs will return a 0
	\item If x $\in$ L, if any of the t runs returns a 1, we return the correct answer.
	\item If x $\in$ L, if all t runs returns 0, we make mistake
	\item Pr[algorithm makes a mistake t times] $\leq \frac{1}{2^{t}}$
	\item Thus, we can make the error exponentially small by polynomial number of repetitions.

\end{itemize}
\end{frame}
%------------------------------------------------------
\subsection{BPP}
\begin{frame}
\frametitle{Bounded-error Probabilistic Polynomial time}
\begin{definition}[BPP]
	The complexity class BPP is the class of all languages L for which there exists a polynomial PTM M such that
	$$
	x \in L \Rightarrow Pr[M(x) = 1] \geq 2/3
	$$
	$$
	x \notin L \Rightarrow Pr[M(x) = 1] \leq 1/3
	$$
	M answers correctly with probability 2/3 on any input x regardless if x $\in$ L or x $\notin$ L
\end{definition}
P $\subseteq$ BPP $\subseteq$ EXP\\
BPP = coBPP\\
\end{frame}

%-------------------------------------------------------
\begin{frame}
\frametitle{Error Reduction for BPP}
\begin{itemize}
	\item 2/3 is arbitrary and can be improved as follows:
	\begin{itemize}
		\item Repeat the algorithm t times, say it returns $X_{i}$ at the i-th run
		\item Take majority answer, i.e., if $\geq$ t/2 times M returns 1, return 1, otherwise return 0

	\end{itemize}
	\item {\color{red} The Chernoff Bound}
	\begin{itemize}
		\item Suppose $X_{1},...,X_{t}$ are t independent random variables with values in $\{0, 1\}$ and for every i, $Pr[X_{i} = 1] = p$. Then
		$$
		Pr[\frac{1}{t}\sum_{i=1}^{t}X_{i}-p > \epsilon] < e^{-\epsilon^{2}\frac{t}{2p(1-p)}}
		$$
		$$
		Pr[\frac{1}{t}\sum_{i=1}^{t}X_{i}-p < -\epsilon] < e^{-\epsilon^{2}\frac{t}{2p(1-p)}}
		$$
	\end{itemize}
\end{itemize}
\end{frame}
%-------------------------------------------------------
\begin{frame}
\frametitle{Error Reduction for BPP}
\begin{itemize}
	\item x $\in$ L, $Pr[X_{i} = 1] = p \geq 2/3$, M outputs correctly if $\sum_{i=1}^{t}X_{i} \geq t/2$
	\item The algorithm makes a mistake when $\sum_{i=1}^{t}X_{i} < t/2$
	\item Pr[Algorithm outputs the wrong answer on x]\\
	$= Pr[\sum_{i=1}^{t}X_{i} < t/2]$\\
	$= Pr[\frac{1}{t}\sum_{i=1}^{t}X_{i} < 1/2]$\\
	$= Pr[\frac{1}{t}\sum_{i=1}^{t}X_{i}-p < 1/2-p]$\\
	$< e^{-(\frac{1}{2}-p)^{2}\frac{t}{2p(1-p)}}$\\
	$\leq e^{-2t(\frac{1}{2}-p)^{2}}$\\
	$\leq e^{-2t(\frac{1}{2}-\frac{2}{3})^{2}}$\\
	$= e^{-ct}$, where c is some constant\\
	$= \frac{1}{2^{n}}$ if we set $t = \frac{n \ln 2}{c} = O(n)$\\
	\item So by running the algorithm O(n) times, reduce probability exponentially.
\end{itemize}
\end{frame}
%-------------------------------------------------------
\begin{frame}
\frametitle{Another Definition of BPP}
\begin{definition}[BPP]
	The complexity class BPP is the class of all languages L for which there exists a polynomial PTM M such that
	$$
	x \in L \Rightarrow Pr[M(x) = 1] \geq 1-\frac{1}{2^{|x|}}
	$$
	$$
	x \notin L \Rightarrow Pr[M(x) = 1] \leq \frac{1}{2^{|x|}}
	$$
	M answers correctly with probability $1-\frac{1}{2^{|x|}}$ on any input x regardless if x $\in$ L or x $\notin$ L
\end{definition}
RP $\cup$ coRP $\subseteq$ BPP.\\
We have showed that $RP \subseteq NP$, But We don't know if $BPP \subseteq NP$
\end{frame}

%-------------------------------------------------------
\subsection{ZPP}
\begin{frame}
\frametitle{Zero-error Probabilistic Polynomial time}
\begin{definition}[ZPP]
	The complexity class ZPP is the class of all languages L for which there exists a polynomial PTM M such that
	$$
	x \in L \Rightarrow M(x)=1\ or\ M(x) =\ 'Don't\ know'
	$$
	$$
	x \notin L \Rightarrow M(x)=0\ or\ M(x) =\ 'Don't\ know'
	$$
	$$
	\forall x, Pr[M(x) = \ 'Don't\ know'] \leq 1/2
	$$
	Whenever M answers with a 0 or a 1, it answers correctly. If M is not sure, it'll output a 'Don't know'. On any input x, it outputs 'Don't know' with probability at most 1/2.
\end{definition}
\end{frame}
%-------------------------------------------------------
\begin{frame}
\frametitle{Another Definition of ZPP}
{\color{red} Algorithm M'}:\\
{\color{red} Input}:x\\
while(1):\\
\begin{itemize}
	\item Invoke M to determine x $\in$ L or not;
	\item If b = M(x), {\color{red} ouput} b and halt;
	\item else continue;
\end{itemize}
If the running time of M is T(|x|), the expecting running time of M is 2T(|x|). 
\begin{definition}[ZPP]
	The complexity class ZPP is the class of all languages L for which there exists a expecting polynomial PTM M such that
	$$
	x \in L \Rightarrow M(x)=1
	$$
	$$
	x \notin L \Rightarrow M(x)=0
	$$
\end{definition}
\end{frame}
%-------------------------------------------------------
\begin{frame}
\frametitle{A Theorem on ZPP}
\begin{theorem}
	ZPP = RP $\cap$ coRP
\end{theorem}
\begin{itemize}
	\item This theorem is surprising, since the corresponding question for nondeterministic(i.e. P = NP $\cap$ coNP) is open.
	\item We have to prove both ways:
	$$
	ZPP \subseteq RP \cap coRP
	$$
	$$
	RP \cap coRP \subseteq ZPP
	$$
\end{itemize}
\end{frame}

%------------------------------------------------

\begin{frame}
\frametitle{First direction}
$$
ZPP \subseteq RP \cap coRP
$$
\begin{itemize}
	\item We prove ZPP $\subseteq$ coRP, the other proof is similar
	\item Let L $\in$ ZPP. Then $\exists$ PTM M that, $\forall$ x, either correctly decides x $\in$ L or outputs 'Don't know'
	\item Let M' be the Turing Machine that on input x, returns 1 if M(x) ='Don't know' and otherwise returns M(x).
	\begin{itemize}
		\item x $\in$ L: M(x) = 1 or M(x) ='Don't know', so M'(x) = 1.

		\item x $\notin$ L: M(x) = 0, which is correct, or M(x) = 'Don't know' where M'(x)  returns incorrectly with probability $\leq$ 1/2
	\end{itemize}
	\item So L $\in$ coRP, and ZPP $\subseteq$ coRP
\end{itemize}
\end{frame}

%------------------------------------------------

\begin{frame}
\frametitle{Other direction}
$$
RP \cap coRP \subseteq ZPP
$$
\begin{itemize}
	\item L $\in$ RP $\cap$ coRP. Then there exist two TM's s.t.:
	\begin{itemize}
		\item $M_{1}(x) = 1$ if x $\in$ L. Incorrect for x $\notin$ L with prob. $\leq$ 1/2
		\item $M_{2}(x) = 0$ if x $\notin$ L. Incorrect for x $\in$ L with prob. $\leq$ 1/2
	\end{itemize}
	\item Construct a PTM M' which works as follows on M'(x):
	\begin{itemize}
		\item Run $M_{1}(x)$, and if $M_{1}(x) = 0$, return $x \notin$ L
		\item Run $M_{2}(x)$, and if $M_{2}(x) = 1$, return $x \in$ L
		\item Else return 'Don't know'
	\end{itemize}
	\item {\color{red} Claim}: If M'(x) returns a 0 or a 1, it is correct.

	\item {\color{red} Claim}: M'(x) returns 'Don't know' with prob. $\leq$ 1/2
	\begin{itemize}
		\item Assume x $\in$ L. 
		\item Pr[M' return 'Don't know'] = Pr[$M_{1}(x)$ = 1 $\land$ $M_{2}(x)$ = 0] \\
		= Pr[$M_{2}(x)$ = 0] $\leq$ 1/2
		\item The case for x $\notin$ L similar 
	\end{itemize}
	\item So, L $\in$ ZPP
\end{itemize}
\end{frame}

%------------------------------------------------
\section{Relationship Between BPP and Ohter Class}
\subsection{BPP is in P/poly}
\begin{frame}
\frametitle{BPP is in P/poly}
\begin{theorem}
	BPP $\subseteq$ P/poly
\end{theorem}
Proof:
\begin{itemize}
	\item L $\in$ BPP. Define L(x), L(x) = 1 if x $\in$ L, else L(x) = 0
	\item $\exists$ A PTM M on input x $\in$ $\{0,1\}^{n}$, use m random bits s.t.\\
	$\forall$ x $\in$ $\{0,1\}^{n}$, Pr[M(x, r) $\neq$ L(x)] $\leq$ $2^{-n-1}$
	\item Say r $\in$ $\{0,1\}^{m}$ is bad for an input x $\in$ $\{0,1\}^{n}$ if M(x, r) $\neq$ L(x), otherwise call r good for x
	\item $\forall$ x, at most $\frac{2^{m}}{2^{n+1}}$ string r are bad for x
	\item Adding over all x $\in$ $\{0,1\}^{n}$ there are at most $2^{n}\frac{2^{m}}{2^{n+1}} = 2^{m}/2$  string r are bad
	\item So, $\exists$ $r_{0} \in \{0,1\}^{m}$ good for every x $\in$ $\{0,1\}^{n}$
	\item We can use $r_{0}$ as a advice, so L $\in$ P/poly
\end{itemize}
\end{frame}

%-----------------------------------------------
\subsection{BPP is in PH}
\begin{frame}
\frametitle{BPP is in PH}
\begin{theorem}
BPP $\subseteq$ $\Sigma_{2} \cap \Pi_{2}$
\end{theorem}
\begin{itemize}
	\item First, we prove BPP $\subseteq \Sigma_{2}$ 
	\item Suppose L $\in$ BPP, then $\exists$ ppt M for L s.t.
	$$
	x \in L \Rightarrow Pr[M(x, r) = 1] \geq 1-2^{-n}
	$$
	$$
	x \notin L \Rightarrow Pr[M(x, r) = 1] \leq 2^{-n}
	$$
	|r| = m = poly(n)
	\item for x $\in$ $\{0,1\}^{n}$, let $S_{x}$ denote the set of r's for which M(x,r) = 1. Then:
	$$
	x \in L \Rightarrow |S_{x}| \geq (1-2^{-n})2^{m}
	$$
	$$
	x \notin L \Rightarrow |S_{x}| \leq (2^{-n})2^{m}
	$$
\end{itemize}
\end{frame}
%------------------------------------------------
\begin{frame}
\frametitle{BPP is in PH}
\begin{itemize}
	\item For a set S $\subseteq \{0, 1\}^{m}$ and a string u $\in \{0, 1\}^{m}$. Define $S+u = \{x+u:x \in S\}$. Let $k = \lceil \frac{m}{n} \rceil +1$, we have the following two claims.
	\item {\color{red} Claim 1}: For every set  S $\subseteq \{0, 1\}^{m}$ with $|S| \leq 2^{m-n}$ and every k vectors $u_{1},...,u_{k}$, $\cup_{i=1}^{k}(S+u_{i}) \neq \{0, 1\}^{m}$
	\item {\color{red} Proof of Claim 1}: 
	\begin{itemize}
		\item Since |S+$u_{i}$| = |S|, $|\cup_{i=1}^{k}(S+u_{i})|$ $\leq$ k|S+$u_{i}$| = k|S| < $2^{m}$(for sufficiently large n)
	\end{itemize}
	\item {\color{red} Claim 2}: For every set S $\subseteq \{0, 1\}^{m}$ with $|S| \geq (1-2^{-n})2^{m}$, there exists $u_{1},...,u_{k}$, s.t. $\cup_{i=1}^{k}(S+u_{i}) = \{0, 1\}^{m}$
\end{itemize}

\end{frame}
%--------------------------------------------------
\begin{frame}
\frametitle{BPP is in PH}
{\color{red} Proof of Claim 2}: 
\begin{itemize}
	\item We claim if $u_{1},...,u_{k}$ are chosen independently at random then Pr[$\cup_{i=1}^{k}(S+u_{i}) = \{0, 1\}^{m}$] > 0 
	\item For r $\in \{0, 1\}^{m}$, let $B_{r}$ denote the "bad event" r $\notin$ $\cup_{i=1}^{k}(S+u_{i})$. It's suffice to show $Pr[\exists r \in \{0, 1\}^{m}\ B_{r}] < 1$
	\item $Pr[\exists r \in \{0, 1\}^{m}\ B_{r}]$ $\leq$ $\sum_{r \in \{0, 1\}^{m}}Pr[B_{r}] \leq 2^{m}max\{Pr[B_{r}]\}$\\
	we should only prove $\forall$ r, $Pr[B_{r}] < 2^{-m}$
	\item Let $B_{r}^{i}$ denote the event r $\notin$ $S+u_{i}$(equivalently, $r+u_{i} \notin S$), then $B_{r} = \cap_{i \in [k]} B_{r}^{i}$\\
	\item $r+u_{i}$ is a uniform element in $\{0,1\}^{m}$, it will be in  S with probability at least 1-$2^{-n}$. So, 
	$$
	Pr[B_{r}^{i}] = Pr[r+u_{i} \notin S] = 1-Pr[r+u_{i} \notin S] \leq 2^{-n}
	$$ 
	\item Since the events $B_{r}^{i}$ are independent, we have
	$$Pr[Br] = Pr[\cap_{i \in [k]} B_{r}^{i}] = Pr[B_{r}^{i}]^{k} \leq 2^{-nk} < 2^{-m}$$ 
\end{itemize}
\end{frame}
%--------------------------------------------------
\begin{frame}
\frametitle{BPP is in PH}

\begin{itemize}
	\item {\color{red} Claim 1}: For every set  S $\subseteq \{0, 1\}^{m}$ with $|S| \leq 2^{m-n}$ and every k vectors $u_{1},...,u_{k}$, $\cup_{i=1}^{k}(S+u_{i}) \neq \{0, 1\}^{m}$
	\item {\color{red} Claim 2}: For every set S $\subseteq \{0, 1\}^{m}$ with $|S| \leq (1-2^{-n})2^{m}$, there exists $u_{1},...,u_{k}$, s.t. $\cup_{i=1}^{k}(S+u_{i}) = \{0, 1\}^{m}$
	\item Together with Claim 1 and 2, we know:
	\begin{itemize}
		\item x $\in$ L $\Rightarrow$ $|S_{x}| \geq (1-2^{-n})2^{m}$ \\
		$\Rightarrow$ $\exists$ $u_{1}, ...,u_{k}$ $\in$ $\{0,1\}^{m}$ $\forall$ r $\in$ $\{0,1\}^{m}$ r $\in$ $\cup_{i=1}^{k}(S_{x}+u_{i})$
		\item x $\notin$ L $\Rightarrow$ $|S_{x}| \leq 2^{m-n}$ \\
		$\Rightarrow$ $\forall$ $u_{1}, ...,u_{k}$ $\in$ $\{0,1\}^{m}$ $\exists$ r $\in$ $\{0,1\}^{m}$ r $\notin$ $\cup_{i=1}^{k}(S_{x}+u_{i})$
		\item Whether r $\in$ $\cup_{i=1}^{k}(S_{x}+u_{i})$ or not can be decided in polynomial time.
	\end{itemize} 
	\item So, L $\in$ $\Sigma_{2}$, BPP $\subseteq$ $\Sigma_{2}$
	\item L $\in$ BPP $\Rightarrow$ $\overline{L} \in coBPP = BPP$ $\Rightarrow$ $\overline{L} \in \Sigma_{2}$ $\Rightarrow$ $L \in co\Sigma_{2} = \Pi_{2}$
	\item So, BPP $\subseteq$ $\Pi_{2}$, BPP $\subseteq$ $\Sigma_{2} \cap \Pi_{2}$
\end{itemize}
\end{frame}
%--------------------------------------------------
\subsection{Unsolved Problem about BPP}
\begin{frame}
\frametitle{Unsolved Problem about BPP}
\begin{itemize}
	\item BPP = P?
	\item Complete problem for BPP?
	\item Does BPTIME have a hierarchy theorem?
\end{itemize}
\end{frame}
\begin{frame}
\Huge{\centerline{The End}}
\end{frame}

%----------------------------------------------------------------------------------------

\end{document} 